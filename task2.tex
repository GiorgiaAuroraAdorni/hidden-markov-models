\Problem{Your friend has two urns, labelled 1 and 2. Urn 1 contains 5 blue, 2 
red and 4 yellow balls. Urn two contains 3 blue, 4 red and 3 yellow balls. She 
covers your eyes with a tape. Then, she chooses one urn at random with equal 
probability. You pick one ball from that urn, she tells you its colour and then 
you return the ball to the urn you picked it from (you don’t know which one as 
your eyes are covered). Your friend switches from urn 1 to urn 2 with 
probability ${1}{}/{2}$ and from urn 2 to urn 1 with probability ${3}{}/{4}$. 
You pick one ball 
again, she tells you its colour and the process repeats.

\begin{itemize}
	\item[(a)] Describe the system as Hidden Markov Model. What are $S, O, \pi, 
	A, B$?
	\item[(b)] What is the probability that initial urn was urn 1, then urn 2 
	and urn 1 again given that you picked yellow, red and blue balls 
	respectively. Use dynamic programming! \textit{Hint}: use Bayes formula!
	\item[(c)] What is the most probable sequence of urns given that you picked 
	red, yellow and blue. Use Viterbi Algorithm!
\end{itemize}
}

\Solution{
	
	\begin{itemize}
		\item[(a)] 
		\begin{minipage}[t]{\linewidth}
		\begin{figure}[H]
		\centering
		\includegraphics[width=0.5\linewidth]{hmm2.pdf}
		\captionof{figure}{Representation of the described system as Hidden 
		Markov Model.}
		\label{fig:hmm}
		\end{figure}
		\end{minipage}
	
		\begin{itemize}
			\item[-] Set of N states $S=\{s_1, \dots, s_N\}$: in this case, the 
			states are the two urns, so 
			\begin{equation*}
				S=\{u_1, u_2\} \mbox{.}
			\end{equation*}
			\item[-] The starting state probabilities are contained in 
			the vector $\pi=\{\pi_1, \dots, \pi_N\}$. In this example, since 
			each urn is chosen at random with equal probability:
			\begin{equation*}
			\pi = 
			\left[\begin{matrix}
			\dfrac{1}{2}\\[15pt] \dfrac{1}{2} 
			\end{matrix}\right]\mbox{.}
			\end{equation*}
			\item[-] State transition probabilities are given in matrix $A$, 
			where the probability of a transitioning from state $i$ to state 
			$j$ is  $a_{ij}=\mathrm{P}(q_{t+1}=s_j|q_t=s_i)$, with $i \geq 1$ 
			and $j \leq N$. In this example, since the probability of switches 
			from urn 1 to urn 2 is ${1}{}/{2}$ and from urn 2 to urn 1 
			is ${3}{}/{3}$, then
			\begin{equation*}
				A = 
				\left[\begin{matrix}
				\dfrac{1}{2}  & \dfrac{1}{2} \\[15pt]
				\dfrac{3}{4}  & \dfrac{1}{4}
				\end{matrix}\right] \mbox{.}
			\end{equation*}
			
			\item[-] The $M$ observations $O=\{ v_1, \dots, v_m\}$ in the 
			space. In this example, the possible observations correspond to 
			picking a blue, red or yellow ball
			\begin{equation*}
				O = \{ b, r, y \} \mbox{.}
			\end{equation*} 
	
			\item[-] Observation probabilities are give in the matrix $B$. 
			In particular, the probability of observing $m$ is given by 
			$b_i(m)=\mathrm{P}(O_t=v_m|q_t=s_i)$, with $1 \leq i \leq N$ and 
			$1 \leq j \leq M$
			
			\begin{equation*}
			B = 
			\left[\begin{matrix}
			\dfrac{5}{11} & \dfrac{2}{11} & \dfrac{4}{11}\\[15pt]
			\dfrac{3}{10} & \dfrac{4}{10} & \dfrac{3}{10} 
			\end{matrix}\right]
			\mbox{.}
			\end{equation*}
			
				
		\end{itemize}
	\end{itemize}
}
